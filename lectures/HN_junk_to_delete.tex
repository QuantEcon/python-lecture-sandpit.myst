
 This schedule was computed by finding the optimal policy functions

$$
\begin{align} V^u_{t+1} & = f(V^u_t) \cr
             c_t & = g(V^u_t)
\end{align}
$$

and iterating on them, starting from some initial $V^u_0 > V_{\rm aut}$,
where $V_{\rm aut}$ is the autarky level for an unemployed worker.
Notice how the replacement ratio declines with duration.
Figure XXX
sets $V^u_0$ at 16,942, a number that
has to be interpreted in the context of Hopenhayn and Nicolini's
parameter settings.

We computed these numbers using the parametric version studied by Hopenhayn
and Nicolini.


 Hopenhayn and Nicolini chose parameterizations and parameters as follows:
They interpreted one  period as one week, which led them
to set $\beta=.999$.  They took  $u(c) = {c^{(1-\sigma)} \over 1 - \sigma}$
and set
$\sigma=.5$.  They set the wage $w=100$ and
specified the hazard function to be  $p(a) = 1 - \exp(-ra)$, with $r$ chosen
to give a hazard rate $ p(a^*) = .1$, where
$a^*$ is the optimal search  effort under autarky. To compute the numbers
in Figure XXXXX we used
these same settings.

+++